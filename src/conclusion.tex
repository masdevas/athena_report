\clearpage
\section{Заключение}
В процессе разработки инструментов работы с искусственным интеллектом были изучены фреймворки для работы с глубокими нейронными сетями Caffe, Tensorflow, Keras, а также библиотеки алгоритмов машинного обучения scikit-learn, oneDAL. Были получены сведения об использовании алгоритмов машинного и глубокого обучения в современном мире. Также были изучены методы проектирования программных продуктов, которые в дальнейшем были использованы в разработке фреймворка.

Описанный фреймворк с помощью удобного программного интерфейса позволяет строить требуемую пользователю нейронную сеть и комбинировать модели машинного обучения. Был реализован пример алгоритма машинного обучения - логистической регрессии, который может быть успешно применен в решении практических задач. Автоматическая генерация градиента позволяет пользователю строить произвольные графы и не думать о том, каким образом они будут оптимизированы - фреймворк берет это на себя. Использование LLVM и подменяемой библиотеки функций позволяет добиться необходимой производительности в вычислении отдельных узлов графа, а возможности LLVM оптимизируют граф как единую функцию. При расширении библиотеки функций и создании множества готовых моделей искусственного интеллекта фреймворк сможет предоставить пользователю большое количество готового к применению функционала для решения прикладных задач искусственного интеллекта.
